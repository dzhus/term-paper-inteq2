\documentclass[11pt]{article}
\usepackage[utf8x]{inputenc}
\usepackage[english,russian]{babel}
\usepackage{amsmath,amssymb}

\usepackage[top=2.5cm, bottom=3cm, left=6cm, right=3.5cm]{geometry}

% rich title
\usepackage{titling}

\numberwithin{equation}{section}

% Russian traditions
\renewcommand{\epsilon}{\varepsilon}
\renewcommand{\phi}{\varphi}
\renewcommand{\leq}{\leqslant}
\renewcommand{\geq}{\geqslant}
\newcommand{\intl}{\int\limits}
\usepackage{misccorr}

% TikZ
\usepackage{tikz}
\usepackage{pgfplots}
\pgfplotsset{every axis grid/.style={style=help lines}}

% Bib in TOC
\usepackage[numbib,nottoc]{tocbibind}

% Custom commands
\renewcommand{\vec}[1]{{}^{\vee}\negmedspace#1}
\newcommand{\at}[2]{\left. {#1}\right\vert_{#2}}
\newcommand{\intat}[3]{\left. {#1}\right\vert^{#2}_{#3}}
\newcommand{\program}[1]{{\tt #1}}
\newcommand{\name}{\textsc}
\newcommand{\scalmult}[1]{{\left \langle #1 \right \rangle}}
\newcommand{\abs}[1]{\left \lvert{#1}\right \rvert}
\newcommand{\norm}[1]{\left \lVert{#1}\right \rVert}
\newcommand{\set}[1]{\mathbb{#1}}
\newcommand{\mul}{\cdot}

\usepackage[unicode,
pdftex, colorlinks, linkcolor=black, citecolor=black,
pdfauthor=Dmitry Dzhus]{hyperref}

% literate programming tool
\usepackage{noweb}
\noweboptions{nomargintag}

\begin{document}

\author{Дмитрий Джус}
\title{Курсовая работа по теме \\
  \Huge{«Интегральные уравнения»}}
\pretitle{\begin{center}\LARGE}
  \posttitle{\par\end{center}\vskip 3pc}
\date{}
\maketitle
\thispagestyle{empty}

\clearpage
\tableofcontents

\clearpage
\section{Предмет работы}
\label{sec:problem}
В настоящей курсовой работе рассматриваются методы решения
симметричного неоднородного интегрального уравнения Фредгольма второго
рода вида
\begin{equation}
  \label{eq:ieqgen}
  \intl_a^b {e^{-s\tau}x(\tau)\,d\tau} = z(s),\, s\in [a;b]
\end{equation}
где $a = 0, b = 2$.

К использованию предлагается модельное решение вида

\begin{equation}
  \label{eq:model-sol}
  x_0(\tau) = 100 \mul \sin^2\left(\frac{\pi}{2}\tau\right)
\end{equation}

\begin{figure}[!htb]
  \centering
  \begin{tikzpicture}
    \begin{axis}[
      xtick={-1, 0, 1},
      ytick = {0, 100},
      color=black]
      \addplot plot[mark=*,smooth,domain=-1:1,samples=20] function{100*sin(pi/2*(x-1))**2};
    \end{axis}
  \end{tikzpicture}
  \caption{Модельное решение \eqref{eq:model-sol}}
  \label{fig:model-sol}
\end{figure}

\section{Решение}

\subsection{Приближение полиномами Чебышева}

Разложим искомую функцию $x(\tau)$ по многочленам Чебышева до $n$-го члена:
\begin{equation}
  \label{eq:cheb-sum}
  x_A(\tau) = \sum_{k=1}^n{c_k T_k(\tau)}.
\end{equation}

Коэффициенты $c_k$ вычисляются по формуле:
\begin{equation}
  \label{eq:cheb-coeffs}
  c_k = \frac{\scalmult{T_k, x}}{\scalmult{T_k, T_k}},
\end{equation}
где
\begin{equation}
  \scalmult{f, g} = \sum_{m=1}^n{f(\tau_m)\mul g(\tau_m)}.
\end{equation}

При этом известно (см. \cite{bahvalov01}, \cite{amosov94}), что для
интерполяции с наименьшей погрешностью на отрезке $[a;b]$ узлы $\tau_m$
необходимо выбирать следующим образом:
\begin{equation}
  \label{eq:cheb-roots}
  \tau_m = \frac{a+b}{2} +
  \frac{b-a}{2}\mul\cos\left(\frac{\pi(2m-1)}{2n}\right),\quad m = 1,\ldots,n.
\end{equation}

При выборе узлов интерполяции как в \eqref{eq:cheb-roots},
\begin{equation}
  \label{eq:cheb-norm}
  \scalmult{T_k, T_k} = r_k =
  \begin{cases}
    n,&k = 1\\
    n/2,&k>1
  \end{cases}.
\end{equation}

С учётом \eqref{eq:cheb-coeffs} и \eqref{eq:cheb-norm} формула
\eqref{eq:cheb-sum} примет вид:
\begin{equation}
  \label{eq:cheb-interpol}
  x_A(\tau) = \sum_{k=1}^n \left[
    \frac{\sum_{j=1}^n 
      x(\tau_j)\mul T_k(\tau_j)}{r_k}
    \mul T_k(\tau)
  \right]
\end{equation}

\subsection{Дискретизация исходного уравнения}
Введём на $[a; b]$ равномерную сетку $S_n = \langle a = s_1, \dotsc,
s_n = b \rangle$. Подставим разложение \eqref{eq:cheb-interpol} в
\eqref{eq:ieqgen} и запишем уравнение для каждого $s_i$:
\begin{equation}
  \label{eq:psi-matrix}
  \intl_a^b
  e^{-s_i\tau}
  \sum_{k=1}^n\sum_{j=1}^n \left[
    x(\tau_j)\mul T_k(\tau_j) \mul \frac{1}{r_k} \mul T_k(\tau)
  \right]\,d\tau = 
  z(s_i).
\end{equation}

Переупорядочим знаки суммирования и интегрирования:
\begin{equation}
  \label{eq:psi-theta}
  \sum_{j=1}^n
    \sum_{k=1}^n \underbrace{
      \intl_a^b e^{-s_i\tau}\mul T_k(\tau)\,d\tau}_{\psi_{ik}}
    \mul \underbrace{\vphantom{\intl_a^b} T_k(\tau_j) \mul \frac{1}{r_k}}_{\theta_{kj}}
  \mul x(\tau_j) = z(s_i),
\end{equation}
с учётом введённых обозначений:
\begin{equation}
  \label{eq:b-coeff}
  \sum_{j=1}^n{ 
    \underbrace{\sum_{k=1}^n \psi_{ik} \mul \theta_{kj}}_{b_{ij}} \mul
    x(\tau_j)} = z(s_i).
\end{equation}

Определим матрицы $\Psi = (\psi_{ik})_n^n$ и $\Theta =
(\theta_{kj})_n^n$, после чего можно ввести матрицу 
\begin{equation}
  \label{eq:b-matrix}
  B = (b_{ij})_n^n = \Psi\mul\Theta.
\end{equation}

Из \eqref{eq:b-coeff} следует:
\begin{equation*}
  \sum_{j=1}^nb_{ij}x(\tau_j) = z(s_i),
\end{equation*}
что с учётом определения \eqref{eq:b-matrix} принимает вид:
\begin{equation}
  \label{eq:discrete}
  B\mul\vec{x}=\vec{z}.
\end{equation}

\clearpage
\section{Информация о документе}

Данный документ был подготовлен с использованием \LaTeX{}. В качестве
реализации языка Scheme использовалась \program{PLT Scheme}.
Иллюстрации созданы с помощью пакета \program{pgfplots} и
\program{gnuplot}.

Автоматизация процесса сборки обеспечивалась утилитами
\program{GNU Make} и \program{texdepend}.

Представленная работа выполнена в рамках программы пятого семестра
обучения по специальности «Вычислительная математика и математическая
физика» в МГТУ им. Н. Э. Баумана.

Дата компиляции настоящего документа: \today
\newcommand{\BibEmph}{\name}
\bibliographystyle{gost71s}
\bibliography{paper}

\end{document}

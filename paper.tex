\documentclass[11pt]{article}
\usepackage[utf8x]{inputenc}
\usepackage[english,russian]{babel}
\usepackage{amsmath,amssymb}

\usepackage[top=2.5cm, bottom=3cm, left=6cm, right=3.5cm]{geometry}

% rich title
\usepackage{titling}

\numberwithin{equation}{section}

% Russian traditions
\renewcommand{\epsilon}{\varepsilon}
\renewcommand{\phi}{\varphi}
\renewcommand{\leq}{\leqslant}
\renewcommand{\geq}{\geqslant}
\newcommand{\intl}{\int\limits}
\usepackage{misccorr}

% TikZ
\usepackage{tikz}
\usepackage{pgfplots}
\pgfplotsset{every axis grid/.style={style=help lines}}

% Bib in TOC
\usepackage[numbib,nottoc]{tocbibind}

% Custom commands
\renewcommand{\vec}[1]{{}^{\vee}\negmedspace#1}
\newcommand{\at}[2]{\left. {#1}\right\vert_{#2}}
\newcommand{\intat}[3]{\left. {#1}\right\vert^{#2}_{#3}}
\newcommand{\program}[1]{{\tt #1}}
\newcommand{\name}{\textsc}
\newcommand{\scalmult}[1]{{\left \langle #1 \right \rangle}}
\newcommand{\abs}[1]{\left \lvert{#1}\right \rvert}
\newcommand{\norm}[1]{\left \lVert{#1}\right \rVert}
\newcommand{\set}[1]{\mathbb{#1}}
\newcommand{\mul}{\cdot}
\newcommand{\figref}[1]{рис. \ref{#1}}

\usepackage[unicode,
pdftex, colorlinks, linkcolor=black, citecolor=black,
pdfauthor=Dmitry Dzhus]{hyperref}

% literate programming tool
\usepackage{noweb}
\noweboptions{nomargintag,smallcode}

\begin{document}

\author{Дмитрий Джус}
\title{Курсовая работа по теме \\
  \Huge{«Интегральные уравнения»}}
\pretitle{\begin{center}\LARGE}
  \posttitle{\par\end{center}\vskip 3pc}
\date{}
\maketitle
\thispagestyle{empty}

\clearpage
\tableofcontents

\clearpage

\input{solution.tex}

\clearpage
\section{Информация о документе}

Данный документ был подготовлен с использованием \LaTeX{}. Программа
из раздела \ref{sec:implementation} написана в среде
\program{GNU Octave}. Код представлен с использованием
\program{noweb}. Иллюстрации созданы с помощью пакета
\program{pgfplots} и \program{gnuplot}.

Автоматизация процесса сборки обеспечивалась утилитами
\program{GNU Make} и \program{texdepend}.

Представленная работа выполнена в рамках программы седьмого семестра
обучения по специальности «Вычислительная математика и математическая
физика» в МГТУ им. Н. Э. Баумана.


Дата компиляции настоящего документа: \today
\newcommand{\BibEmph}{\name}
\bibliographystyle{gost71s}
\bibliography{paper}

\end{document}

\documentclass[11pt]{article}
\usepackage[utf8x]{inputenc}
\usepackage[english,russian]{babel}
\usepackage{amsmath,amssymb}

\usepackage[top=2.5cm, bottom=3cm, left=6cm, right=3.5cm]{geometry}

% rich title
\usepackage{titling}

\numberwithin{equation}{section}

% Russian traditions
\renewcommand{\epsilon}{\varepsilon}
\renewcommand{\phi}{\varphi}
\renewcommand{\leq}{\leqslant}
\renewcommand{\geq}{\geqslant}
\newcommand{\intl}{\int\limits}
\usepackage{misccorr}

% TikZ
\usepackage{tikz}
\usepackage{pgfplots}
\pgfplotsset{every axis grid/.style={style=help lines}}

% Bib in TOC
\usepackage[numbib,nottoc]{tocbibind}

% Custom commands
\renewcommand{\vec}[1]{{}^{\vee}\negmedspace#1}
\newcommand{\at}[2]{\left. {#1}\right\vert_{#2}}
\newcommand{\intat}[3]{\left. {#1}\right\vert^{#2}_{#3}}
\newcommand{\program}[1]{{\tt #1}}
\newcommand{\name}{\textsc}
\newcommand{\scalmult}[1]{{\left \langle #1 \right \rangle}}
\newcommand{\abs}[1]{\left \lvert{#1}\right \rvert}
\newcommand{\norm}[1]{\left \lVert{#1}\right \rVert}
\newcommand{\set}[1]{\mathbb{#1}}
\newcommand{\mul}{\cdot}
\newcommand{\figref}[1]{рис. \ref{#1}}

\usepackage[unicode,
pdftex, colorlinks, linkcolor=black, citecolor=black,
pdfauthor=Dmitry Dzhus]{hyperref}

% literate programming tool
\usepackage{noweb}
\noweboptions{nomargintag}

\begin{document}

\author{Дмитрий Джус}
\title{Курсовая работа по теме \\
  \Huge{«Интегральные уравнения»}}
\pretitle{\begin{center}\LARGE}
  \posttitle{\par\end{center}\vskip 3pc}
\date{}
\maketitle
\thispagestyle{empty}

\clearpage
\tableofcontents

\clearpage

\input{solution.oct.tex}

\subsection{Устойчивость}

Внесём в правую часть уравнения малое возмущение порядка $10^{-2}$,
рассмотрев вместо \eqref{eq:discrete} следующую систему:
\begin{equation}
  \label{eq:discrete-dist}
  B\mul\vec{x}=\vec{\tilde{z}}.
\end{equation}

Решение данной системы $\vec{\tilde{x}}$ (\figref{fig:raw-dist})
демонстрирует её неусточивость к малым возмущениям.

\begin{figure}[htb]
  \centering
  \begin{tikzpicture}
    \begin{axis}
      \input{raw-dist-sol-plot.tex}
    \end{axis}
  \end{tikzpicture}
  \caption{Решение \eqref{eq:discrete} после внесения возмущения}
  \label{fig:raw-dist}
\end{figure}

\subsection{Регуляризация}

Рассмотрим вместо \eqref{eq:discrete-dist} следующую задачу:
\begin{equation}
  \label{eq:discrete-reg}
  \left(\alpha H + B^TB \right) \mul \vec{x} = B^T\mul \vec{\tilde{z}},
\end{equation}
где $H$ — матрица стабилизирующего функционала, а $\alpha$ — параметр
регуляризации. Обозначая решение новой системы как
$\vec{\bar{x}}(\alpha)$, зададимся задачей минимизации функционала
\begin{equation}
  \Omega(\alpha) = \norm{\vec{x_0} - \vec{\bar{x}}(\alpha)}.
\end{equation}

\begin{figure}[htb]
  \centering
  \begin{tikzpicture}
    \begin{axis}
      \input{w-plot.tex}
    \end{axis}
  \end{tikzpicture}
  \caption{Значения функционала $\Omega(\alpha)$}
\end{figure}

Находя $\alpha$: \input{alpha-val.out}, вычислим решение
регуляризованной системы \eqref{eq:discrete-reg}. Оно сопоставлено с
модельным решением $x_0$ на \figref{fig:r1}.

\begin{figure}[htb]
  \centering
  \begin{tikzpicture}
    \begin{axis}
      \input{model-plot.tex}
      \input{r1-sol-plot.tex}
    \end{axis}
  \end{tikzpicture}
  \caption{Модельное и регуляризационное решение}
  \label{fig:r1}
\end{figure}

Система \eqref{eq:discrete-reg} при найденном значении $\alpha$
обладает устойчивостью: как продемонстрировано на
\figref{fig:r1-dist}, добавление к вектору $\vec{\tilde{z}}$ малого
возмущение уже не приводит к картине, виденной ранее
(\figref{fig:raw-dist}).

\begin{figure}[htb]
  \centering
  \begin{tikzpicture}
    \begin{axis}
      \input{r1-sol-plot.tex}
      \input{r1-sol-dist-plot.tex}
    \end{axis}
  \end{tikzpicture}
  \caption{Стабильность регуляризационного решения $\vec{\bar{x}}(\alpha)$}
  \label{fig:r1-dist}
\end{figure}

Рассматривая для той же системы \eqref{eq:discrete-reg} функционал
\begin{equation}
  \Phi(\beta) = \norm{B\mul\vec{\bar{x}}(\beta) - \vec{\tilde{z}}},
\end{equation}
определим $\beta$: \input{beta-val.out}.

\begin{figure}[htb]
  \centering
  \begin{tikzpicture}
    \begin{axis}
      \input{u-plot.tex}
    \end{axis}
  \end{tikzpicture}
  \caption{Значения функционала $\Phi(\beta)$}
\end{figure}

Решения $\vec{\bar{x}}(\alpha)$ и $\vec{\bar{x}}(\beta)$ представлены
на \figref{fig:r1-r2}.

\begin{figure}[htb]
  \centering
  \begin{tikzpicture}
    \begin{axis}
      \input{r2-sol-plot.tex}
      \input{r2-sol-dist-plot.tex}
    \end{axis}
  \end{tikzpicture}
  \caption{Стабильность регуляризационного решения $\vec{\bar{x}}(\beta)$}
  \label{fig:r2-dist}
\end{figure}

\begin{figure}[htb]
  \centering
  \begin{tikzpicture}[scale=1.5]
    \begin{axis}
      \input{r1-sol-plot.tex}
      \input{r2-sol-plot.tex}
    \end{axis}
  \end{tikzpicture}
  \caption{Сравнение регуляризационных решений}
  \label{fig:r1-r2}
\end{figure}

\clearpage
\section{Информация о документе}

Данный документ был подготовлен с использованием \LaTeX{}. Программа
из раздела \ref{sec:implementation} написана в среде
\program{GNU Octave}. Код представлен с использованием
\program{noweb}. Иллюстрации созданы с помощью пакета
\program{pgfplots} и \program{gnuplot}.

Автоматизация процесса сборки обеспечивалась утилитами
\program{GNU Make} и \program{texdepend}.

Представленная работа выполнена в рамках программы седьмого семестра
обучения по специальности «Вычислительная математика и математическая
физика» в МГТУ им. Н. Э. Баумана.

Дата компиляции настоящего документа: \today
\newcommand{\BibEmph}{\name}
\bibliographystyle{gost71s}
\bibliography{paper}

\end{document}
